% !TEX program = xelatex

\documentclass{resume}
\usepackage[colorlinks,linkcolor=blue,anchorcolor=blue,citecolor=green,urlcolor=blue]{hyperref}

\begin{document}
\pagenumbering{gobble} % suppress displaying page number

\name{Yihao Feng}


\contactInfo{\href{mailto:lewisfyh2012@gmail.com}{lewisfyh2012@gmail.com}}{(+86) 156-529-54818}{}
%keep the last empty braces!

% 
%Education------------- 
% 
\section{Education}
\datedsubsection{\textbf{Beihang University (BUAA)}, Beijng, China}{Aug.2012 --Jul.2016 (Expected)}

\textit{B.E.} in Softerware Engineering

Rank: 4 / 134 \ \  Overall GPA:3.74 / 4 (Major GPA:3.85 / 4)


%
%Programming Skills
%


\section{Programming Skills}
\begin{itemize}[parsep=0.5ex]
  \item Programming Languages: C/C++, Java, C\#, PHP, Python, JavaScript, 
  \item Programming Framework :  Node.js, Django, Android, WPF
  \item Scientiific Software \& System: Matlab, Mathematics, Weka, Hadoop, openSMILE, Torch, LightSide
  \item Database: MySQL, Microsoft SQL Server, MongoDB
\end{itemize}


%
%Experience----------
%

\section{Experience \& Research}
\datedsubsection{\textbf{ArticuLab}, Carnegie Mellon University, USA}{Jul.2015 -- Sept.2015}
\role{Summer Research Intern}{Adviser: Prof.Justine Cassell \& Dr.Alexandros Papangelis}
\textbf{Research Topic}:  Modelling rapport in human or human-virtual peer conversation in a multimodal way
\begin{itemize}
  \item Extracted visual and acoustic features by using openSMILE and CLM-Framework and synthesized them.
  \item Dealt with cleaning data, feature selection, imbalanced data problems and ran baseline  with Weka.
  \item Applied linear-CRF to model rapport in a temporal way and achieved a result of 74\% accuracy.
  \item Tried to capture inner correlation of rapport by using HCRF and achieved a result of 70\% accuracy.
\end{itemize}

\datedsubsection{\textbf{State Key Laboratory of Machine Perception}, Peking University, China}{Oct.2015 -- Present}
\role{Undergraduate Research Assistant}{Adviser: Prof.Guojie Song}
\textbf{Research Topic}: Inferring social relations among users in mobile social network
\begin{itemize}
\item  Proposed a method to extract geographical context meaning from LBS service based on \textbf{TF-IDF}.
\item Ameliorated imbalanced problem by applying a distribution-sensitive learning parameter in graphs.
\item Built a Triadic factor in the model by adopting the Structural Balance Theory in our mobile social network.
\item Built a model called BTFG(Balanced Triadic Factor Graph) based on factor graph and bellowing discovery.
\end{itemize}

\datedsubsection{\textbf{Mathematical Modeling \& Algorithms Laboratory},University of Tsukuba, Japan}{Jul.2014}
\role{Acadamic Visitor}{Adviser: Dr.Yasuyuki Maeda}
\textbf{Research Topic}: Basic research realization in Image Recognition
\begin{itemize}
\item  Learned Basic Machine Learning method(SVM, PCA) to solve Image Recognition problem.
\item Applied the PCA,SVD method to Recognize simple handwriting numbers by using Matlab.
\end{itemize}



\datedsubsection{\textbf{Mobile \& Big Data Department}, Zhiquan.Inc, China}{Feb.2015-Jun.2015}
\role{Software Engineer}{Adviser: Dr.Yun Lu}

\textbf{Project}:Zhiquan Android App Platform
\begin{itemize}
\item Built the IM chat and social relation part of the Android App.
\item Implemented the initial version of User Recommendation System based on \textbf{User-Based} algorithms.
\end{itemize}

%
%% ---selected CURRICULUM DESIGN
%
\section{Selected Curriculum Design}

\datedsubsection{\textbf{Project}: \textit{Online Hospital Booking System}}{Oct.2014-Jan.2015}
\role{Project Manager \& Back-end Developer}{Adviser: Dr.Huobin Tan}
\begin{itemize}
\item Designed the system architeture and set up the system development \& operating enviorment.
\item Designed and built the collections of nosql Database MongoDB.
\item Realized and tested the back-end of the system using NodeJS.
\end{itemize}


\datedsubsection{\textbf{Project}: \textit{Android Smartphone Assistant}}{Mar.2014-Jun.2014}
\role{Android Developer}{Adviser: Prof.Yunxiang Lu}
\begin{itemize}
\item Analyzed project requirements with software engineering methodologies.
\item Designed algorithms of trash cleaning and implemented the method of app locker.
\item Developed the Android app with another student.
\end{itemize}


\datedsubsection{\textbf{Project}: \textit{Android App of Car Pooling}}{Mar.2014-Jun.2014}
\role{Android \& Back-end Developer}{Adviser: Sir.Jun Lin}
\begin{itemize}
\item Designed the system with Agile methodology.
\item Designed algorithms of how to find the suitable car for users.
\item implemented the server of the system using LAMP.
\end{itemize}

%\datedsubsection{\textbf{xxx Projects}}{Jan. 2015 -- Present}
%\role{C, Python, Django, Linux}{Individual Projects, collaborated with xxx}
%Brief introduction: xxx
%\begin{itemize}
%  \item Implemented xxx feature
%%  \item Optimized xxx 5\%
%  \item xxx
%\end{itemize}



% Reference Test
%\datedsubsection{\textbf{Paper Title\cite{zaharia2012resilient}}}{May. 2015}
%An xxx optimized for xxx\cite{verma2015large}
%\begin{itemize}
%  \item main contribution
%\end{itemize}


%
%---Honors & Awards
%
\section{Honors and Awards}
\begin{itemize}

\item \datedline{National Scholarship For Excellent Academic Performance (Top \textbf{2\%})}{Sept.2013}
\item \datedline{National Endeavor Scholarship (Top \textbf{5\%})}{Sept.2014}
\item \datedline{Excellent Student, Beihang University(Top \textbf{5\%}, continued 2 years)}{Sept.2012-Sept.2014}
\item \datedline{\textit{\nth{2} Prize}, Mathematics Competitions of Beihang University}{Jun.2013}
\item \datedline{\textit{\nth{1} Prize of Beijing}, China Undergraduate Mathematical Contest in Modeling}{Sept.2014}

\item \datedline{\textit{\textbf{Meritorious} winner}, Mathematical Contest in Modeling (Top \textbf{13\%})}{Apr.2015}
\end{itemize}


%
%% Publication in the future.
%

%\section {Publication}
%Something 

%
\section{Miscellaneous}
\begin{itemize}[parsep=0.5ex]
\item \textbf{Area of Interests}:Probabilistic Graphical Models, Social Network, Multimodal, Data Mining
\item \textbf{English}: \textit{CET-4}: 609 / 720 \hspace{0.5cm} \textit{CET-6}: 584 / 720 \hspace{0.5cm} \textit{TOFEL}: 96 / 120
\item \textbf{Persoonal Website}:  \url{http://lewiskit.github.io}
\item \textbf{Github}: \url{https://github.com/lewisKit}
\item \textbf{LinkedIn}: \url{https://cn.linkedin.com/in/yihao-feng-15937488}


\end{itemize}

\end{document}
